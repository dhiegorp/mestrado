% -----
% ARQUIVO: capitulo-04.tex
% VERSÃO: 1.1
% DATA: Janeiro de 2016
%
% CAPÍTULO DE METODOLOGIA DA PROPOSTA
%
% NÃO MEXA NAS SEÇÕES, SOMENTE EDITE O CONTEÚDO.
% -----

\chapter{A Proposta}
% #TXT_INTROPROPOSTA
\lipsum[1]

\section{Objetivos e Quest\~{o}es de Pesquisa}
% #TXT_OBJETIVO
\lipsum[1-2]

% PARTE DE QUESTÕES DE PESQUSA - TANTAS QUANTO NECESSÁRIO
\begin{qpesq}
Lorem ipsum dolor sit amet, consectetur adipiscing elit, sed do eiusmod tempor incididunt ut labore et dolore magna aliqua?
\end{qpesq}

\subsection{Hip\'{o}teses}
% #TXT_HIPOTESE
\lipsum[1]

% PARTE DE HIPÓTESES DE PESQUSA - TANTAS QUANTO NECESSÁRIO
\begin{hipo}
Lorem ipsum dolor sit amet, consectetur adipiscing elit, sed do eiusmod tempor incididunt ut labore et dolore magna aliqua?
\end{hipo}

\section{Contribui\c{c}\~{o}es Esperadas}
% #TXT_CONTRIBUICAO
As contribui\c{c}\~{o}es esperadas para este trabalho s\~{a}o:

\begin{enumerate}[label=(\roman*)]
\item Neque porro quisquam est, qui dolorem ipsum quia dolor sit amet, consectetur, adipisci velit, sed quia non numquam eius modi tempora incidunt ut labore et dolore magnam aliquam quaerat voluptatem.

\item At vero eos et accusamus et iusto odio dignissimos ducimus qui blanditiis praesentium voluptatum deleniti atque corrupti quos dolores et quas molestias excepturi sint occaecati cupiditate non provident, similique sunt in culpa qui officia deserunt mollitia animi, id est laborum et dolorum fuga.
\end{enumerate}

