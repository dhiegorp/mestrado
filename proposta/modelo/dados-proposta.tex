%%
%
% ARQUIVO: dados-proposta.tex
%
% VERSÃO: 1.1
% DATA: Janeiro de 2016
% AUTOR: Coordenação PPgSC
% 
%  Arquivo tex com os dados acerca da Proposta de Dissertação.
%
%
%%

%%% AUTOR DA PROPOSTA DE DISSERTAÇÃO (Nome completo)
\autor{Seu Nome Completo}

%%% CÓDIGO DO AUTOR DA PROPOSTA DE DISSERTAÇÃO
\codigoautor{SC XXXXX}

%%% POSTO DO AUTOR DA PROPOSTA DE DISSERTAÇÃO
% ---
%  se o autor é CIVIL, REMOVA ESTA LINHA
%  se o autor é MILITAR, coloque CORRETAMENTE o POSTO aqui
% ---
%\postoautor{1 Ten}

%%% TITULO DA PROPOSTA DISSERTAÇÃO
\titulo{Título Completo da Proposta de Dissertação}

%%% DATA DA APRESENTAÇÃO (formato {dd}{Mmmmm}{aaaa})
\dataapresentacao{31}{Fevereiro}{2016}

%%% ÁREA DE CONCENTRAÇÃO DA PROPOSTA DISSERTAÇÃO
% ---
%  VER SITE do PPGSC para ver as Áreas de Concentração do Programa.
%  Em 2016 só existe uma Área de Concentração:
%     Ciência da Computação
% ---
\area{Área de Concentração da Proposta de Dissertação}

%%% LINHA DE PESQUISA DA PROPOSTA DISSERTAÇÃO
% ---
%  VER SITE do PPGSC para ver as Linhas de Pesquisa do Programa.
%  Em 2016 só existem três Linhas de Pesquisa:
%     Metodologia da Computação
%     Sistemas de Computação
%     Engenharia de Sistemas e Informação
% ---
\linha{Linha de Pesquisa da Proposta de Dissertação}

%%% ORIENTADOR DA PROPOSTA DE DISSERTAÇÃO
% ---
%  CAMPO 1: Nome completo
%  CAMPO 2: D (para D.Sc.); P (para Ph.D.); ou qualquer coisa (inclusive VAZIO) - o que for escrito aparecerá no documento
% ---
\orientador{Nome Completo do Orientador}{D}

%%% POSTO DO ORIENTADOR DA PROPOSTA DE DISSERTAÇÃO
% ---
%  se o orientador é CIVIL, REMOVA ESTA LINHA
%  se o orientador é MILITAR, coloque CORRETAMENTE o POSTO aqui:
%     pode ser: 1 Ten; Cap; Maj; Ten Cel; Cel
% ---
\postoorientador{Ten Cel}

%%% CO-ORIENTADOR DA PROPOSTA DE DISSERTAÇÃO
% ---
%  se NÃO HOUVER co-orientador, REMOVA ESTA LINHA
%  preenchimento idêntico a \orientador{}{}
% ---
%\coorientador{Nome Completo do Co-orientador}{P}

%%% POSTO DO CO-ORIENTADOR DA PROPOSTA DE DISSERTAÇÃO
% ---
%  se NÃO HOUVER co-orientador, REMOVA ESTA LINHA
%  caso contrário
%    se o co-orientador é CIVIL, REMOVA ESTA LINHA
%    se o co-orientador é MILITAR, coloque CORRETAMENTE o POSTO aqui:
%       pode ser: 1 Ten; Cap; Maj; Ten Cel; Cel
% ---
%\postocoorientador{Cap}

%%% COORDENADOR DE PÓS-GRADUAÇÃO
% ---
%  CAMPO 1: Nome completo
%  CAMPO 2: D (para D.Sc.); P (para Ph.D.); ou qualquer coisa (inclusive VAZIO) - o que for escrito aparecerá no documento
% ---
\coord{Nome Completo do Coordenador}{P}

%%% POSTO DO COORDENADOR DE PÓS-GRADUAÇÃO
% ---
%  se o Coordenador de PG é CIVIL, REMOVA ESTA LINHA
%  se o Coordenador de PG é MILITAR, coloque CORRETAMENTE o POSTO aqui:
%     pode ser: 1 Ten; Cap; Maj; Ten Cel; Cel
% ---
\postocoord{Cel}

%%% CHEFE DA SEÇÃO (SE/8)
\chefe{Nome Completo do Chefe da SE/8}

%%% POSTO DO CHEFE DA SEÇÃO (SE/8)
% ---
%  se o Chefe é CIVIL, REMOVA ESTA LINHA
%  se o Chefe é MILITAR, coloque CORRETAMENTE o POSTO aqui:
%     pode ser: 1 Ten; Cap; Maj; Ten Cel; Cel
% ---
\postochefe{Cel}
